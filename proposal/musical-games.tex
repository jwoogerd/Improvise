\documentclass{proc}
\usepackage{verbatim}
\usepackage{biblatex}
\bibliography{bibliography}

\begin{document}

\title{Musical Games: Exploring Improvisational Composition Through Game
Theory}

\author{Caroline Marcks, Andrew Mendelsohn, Jayme Woogerd}

\maketitle

\section{Introduction}

In \emph{A Model of Performance, Interaction, and Improvisation} \cite{hudakberger95}, Paul Hudak
outlines a formal model of music performance and improvisation based on the idea of  \emph{mutually recursive processes}.  

This model lends itself to an form of algorithmic composition using game theory.  In this model, we treat engaged processes as players in a game.  The payoff for each player is quantified by a player's notion of aesthetically pleasing music.  The rules of the game specify the allowable moves of a player given all possible game situations.  Finally, a player's strategy is a (non-deterministic) algorithm for playing the game.

We plan to implement a simple, two-player musical game using the embedded language Hagl, a domain-specific language for defining and exploring game theoretic experiments.  We will use infrastructure provided by the Haskore module to define what is meant by a well-formed move in a musical game.  In general, these moves will be time-stamped musical events.

\section{One-sentence description}

This project will model and implement interactive music composition as a two-player cooperative game.

\section{Project Type}

Originality (software artifact)

\section{Next Steps}

\begin{enumerate}
\item Download and become familiar with Haskore/Euterpea and Hagl.  Experiment with music generation using Paul Hudak's generative grammars and tools.

\item Produce a well-defined (coded) definition for moves (i.e. sequence of time-stamped musical events).  In this case, r is a sequence of time-stamped musical events, which is defined through mutual recursion given a player's interpretation of a score and the most recent moves made by each player.

\textbf{r$_{1}$ = instru$_{1}$ (player$_{1}$ s$_{1}$ r$_{1}$ r$_{2}$)} 

\textbf{r$_{2}$ = instru$_{2}$ (player$_{2}$ s$_{2}$ r$_{2}$ r$_{1}$)}  \\

 To simplify, we will treat the score as a fixed length set of empty measures.

\item Produce a well-defined (coded) definition for a (Hagl) Game that models a simple, two-player game.

\item Using domain knowledge in music theory define a simple quantification scheme for the payoff of a move (i.e. quantify musical aesthetic).

\item Develop a few basic strategies for players.  For example, try to match the note of the other player, try to create the best harmony, try to play dissonance, etc.

\item Experiment with payoffs and strategies to find those that produce nice sounding music.

\end{enumerate}

\section{Related Publications}
\begin{quote}
\textit{List major publications that are most relevant to this project, and how they are related.}
\end{quote}

\begin{enumerate} 
\item \fullcite{hudakberger95}
\item \fullcite{haskore}
\item \fullcite{gametheory}
\item \fullcite{gametheorythesis}
\item \fullcite{HSoM}
\end{enumerate}

\begin{comment}
%\section{Audience} 
%\begin{quote}
%\textit {Who is the audience for this project? 
%How does it meet their needs? 
%What happens if their needs remain unmet?}
%\end{quote}

%Who's going to read this paper?

\section{Approach}
\begin{quote}
\textit {What is your approach and why do you think it's cool and will be successful?}
\end{quote}

List the techniques of the project e.g.
The three steps of the project are:
\begin{enumerate}
\item First technique/approach
\item Second technique/approach
\end{enumerate}

\section{Best-case Impact Statement}
\begin{quote}
\textit {In the best-case scenario, what would be the impact statement (conclusion statement) for this project?}
\end{quote}

What are we looking to achieve?  What's the conclusion?


\section{Obstacles}

\subsection{Major obstacles} % (if these fail, the project is over)

\begin{enumerate} 
\item Possible lack of domain knowledge in music theory, game theory, or both
\item 
\end{enumerate}

\subsection{Minor obstacles}

\begin{enumerate} 
\item One minor obstacle
\item Another minor obstacle
\end{enumerate}

\section{Resources Needed}
\begin{quote}
\textit{What additional resources do you need to complete this project?}
\end{quote}


\section{Define Success}
\begin{quote}
\textit{When / How do you know if you have succeeded in this project?}
\end{quote}
\end{comment}

%\bibliographystyle{abbrv}
\end{document}
