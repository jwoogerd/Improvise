\documentclass{proc}

\begin{document}

\title{Musical Games: Exploring Music Improvisational Composition Through Game
Theory}

\author{Caroline Marcks, Jayme Woogerd, ...?}

\maketitle

\section{Introduction}

In \emph{A Model of Performance, Interaction, and Improvisation}, Paul Hudak
outlines a formal model of music performance and improvisation based on the idea of  \emph{mutually recursive processes}.  This model lends itself to an form of algorithmic composition by an application of game theory - treating engaged processes as players in a game where the currency is manifested as musical aesthetic, the rules of the game specify the allowable moves of a player given all possible game situations, and a player's strategy is a (non-deterministic) algorithm for playing the game.  

We plan to implement a simple, two-player musical game using the embedded language Hagl, a domain-specific language for defining and exploring game theoretic experiments. 

\section{One-sentence description}

Model and implement interactive music composition as a two-player cooperative game.

\section{Project Type}

Originality + Technical depth 

%\section{Audience} 
%\begin{quote}
%\textit {Who is the audience for this project? 
%How does it meet their needs? 
%What happens if their needs remain unmet?}
%\end{quote}

%Who's going to read this paper?

\section{Approach}
\begin{quote}
\textit {What is your approach and why do you think it's cool and will be successful?}
\end{quote}

List the techniques of the project e.g.
The three steps of the project are:
\begin{enumerate}
\item First technique/approach
\item Second technique/approach
\end{enumerate}

\section{Best-case Impact Statement}
\begin{quote}
\textit {In the best-case scenario, what would be the impact statement (conclusion statement) for this project?}
\end{quote}

What are we looking to achieve?  What's the conclusion?

\section{Major Milestones}

\begin{enumerate}
\item Become familiar with Haskore and Hagl -- experiment with music generation 
\item Have a well-defined (coded) definition for moves (i.e. sequence of time-stamped musical events).  

\textbf{r$_{1}$ = instru$_{1}$ (player$_{1}$ s$_{1}$ r$_{1}$ r$_{2}$)} 

\textbf{r$_{2}$ = instru$_{2}$ (player$_{2}$ s$_{2}$ r$_{2}$ r$_{1}$)} 
\item Have some sort of quantification of payoff; quantify musical aesthetic 
\item Define simple strategies for play  -- e.g. try to match the other player exactly, play a melody/accompaniment..other more music theoretically sound strategies for composition
\end{enumerate}


\section{Obstacles}

\subsection{Major obstacles} % (if these fail, the project is over)

\begin{enumerate} 
\item Possible lack of domain knowledge in music theory, game theory, or both
\item 
\end{enumerate}

\subsection{Minor obstacles}

\begin{enumerate} 
\item One minor obstacle
\item Another minor obstacle
\end{enumerate}

\section{Resources Needed}
\begin{quote}
\textit{What additional resources do you need to complete this project?}
\end{quote}


\section{5 Related Publications}
\begin{quote}
\textit{List 5 major publications that are most relevant to this project, and how they are related.}
\end{quote}

\begin{enumerate} 
\item One relevant source
\item Another source
\end{enumerate}

\section{Define Success}
\begin{quote}
\textit{When / How do you know if you have succeeded in this project?}
\end{quote}

\bibliographystyle{abbrv}
\bibliography{proposal-template}
\end{document}
