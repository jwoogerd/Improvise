\documentclass{article}

\usepackage{parskip}

\begin{document}

\section{Introduction}
    What problem did we set out to solve?
    \begin{enumerate}
        \item Modeling interplay between musicians
        \item Build a tool to help people generate music this way
        \item Another thesis statement?
    \end{enumerate}
\section{Background}
\subsection{Music and Improvisation}
    In *A Model of Performance, Interaction, and Improvisation*, Paul Hudak outlines a formal model of musical performance, interaction and improvisation based on the idea that a full musical performance can be understood as a set of complex, interrelated interactions.  

    To motivate this model, he observed that in any performance there exists an interaction between a player and himself: throughout the performance, a good musician will continuously make adjustments to his playing based on what he is hearing from his own instrument. Likewise, in an ensemble or orchestral performance, a player is necessarily affected by the performance of each of the other players in the ensemble.  Finally, each player is working off his or her interpretation of the musical score and, in the case where the players are improvising, the produced performance may actually deviate wildly from what is given in the score.

    In the paper, these interactions are termed *mutually recursive processes* where "the recursion captures feedback [and] mutual recursion captures the interaction between players..."

    In concrete terms, we define these relationships of interaction algebraically:

    \begin{verbatim}
        r = instr(player s r)
    \end{verbatim}

    where *r* is the realization, the music actually produced by a player, and *s* is the score. 

    With this general model for performance and interaction in mind, Hudak asks us to further consider improvisational settings, where he notes, "although the goal of those involved in improvisation in generally one of cooperation, there is also a certain amount of conflict."  Given this natural relationship of conflict and cooperation between players in an improvisational scenario and the mathematical model we can use to frame it, he suggests the application of *game theory* where "engaged processes can be viewed as players in a game, where currency is manifested as aspects of musical aesthetics, and the rules relate to control of such aesthetics.�

\subsection{Game Theory}
    Game theory is the study of strategic decision making.  It is used to model interactions between agents whose decisions affect each other's welfare.

    Before one can formally model a game, a number of things must be defined in the context of that game.  The word "game" refers not to a something a group of kids play at recess, but to any situation with more than two decision-makers.  Those decision-makers are called the players, and there must be a set of rules that discuss the set of legal moves each player can make.  Those each move must be defined in terms of how it effects the state of the game.  In picking what moves to make, each player must adhere their own fixed and well defined strategy that can be algebraically formalized. There must be strict definitions for what information is available in the game to be used by the strategies, both shared amongst the players and not.  Separately from payoff, there must be a measure of success for each player by the end of the game, called that player's payoff.  All of these things together lead to a well defined game that can be modeled and reasoned about.

    Much of the use of game theory is centered around the development of sound and optimal strategies.  An optimal strategy is one that will always lead the player to their best possible payoff.  But how can we begin to reason about the process of reaching such a payoff? The answer lies in game trees.  Game trees represent every possible sequence of moves from a starting game state to an ending game state.  The root of the game tree is the starting state, and each branch represents a possible move for a player or chance from an external event to change the game state.  The leaf nodes represent the "game over" states and have a payoff for each player associated with them.  The best strategies are those that themselves have a concept of a game tree, and traverse it in such a way that they know they will reach optimal payoff nodes.  

    How does all of this fit into the context of a real game?  Let us consider a formal treatment of the game TicTacToe.  The decision makers in TicTacToe are the characters 'X', and 'O'.  They must alternate moves, and place their mark on a 3 by 3 grid, in a previously unoccupied cell.  When there are three cells in a row occupied by the same player (horizontally, vertically, or diagonally), that player wins and the game is over.  If the board fills up before this can happen, there is a draw.  The players share the knowledge of where they have each moved on the board. but nothing further.  The payoffs are fixed to be a 1 for the winning player, -1 for the losing player, and 0 for each player in the case of a draw.  Now let us see what this might look like in the first few levels of the game tree:

    --TicTacToe visual here

    As you can see, the first row models player X making a move, and O's moves in the second row branch off from X's.  In extensive games where players each make multiple moves, these game trees can grow very rapidly.  TicTacToe, for example has 25,000+ nodes in it's game tree, even when eliminating nodes where we can through rotational symmetry.

    But how do game trees work when more than one player must make a decision at a time?  This is where it is necessary to have strict and defined knowledge boundaries for the players.  Simultaneous games are modeled as a layer of strictness on top of alternating games, like TicTacToe.  Players must still make moves one by one, but the information of that choice does not get shared until all players have made their choices.

    --should I go into RPS example here?

    --possible transition paragraph here.

\subsection{Hagl}
    Traditionally, game theory has been used to mathematically determine the best course of action for any given situation.  However, the best course of action is often the selfish one, and requires much computational effort to calculate on the fly.  A subdivision of game theory, called experimental game theory, examines games in the light of competitive experimentation.  This is particularly relevant to games where sub-optimal strategies might yield a better net result.  In Improvise, we are interested in a yielding music that is good for both players, rather than just one at the cost of the other.  

    Hagl is a DSL embedded in Haskell that allows for easy and modular definitions of games.  In Hagl, a game is an instance of the Game type class:

    \begin{verbatim}
        class GameTree (TreeType g) => Game g where
    		type TreeType g :: * -> * -> *
    		type State g
    		type Move g
    		gameTree :: g -> (TreeType g) (State g) (Move g)
    \end{verbatim}

    TreeType, State, and Move are associated types that must be defined in terms of the game.  The only function that must be provided is the gameTree function, that takes a game, and builds it's whole game tree.  Hagl provides a number of operations to then interact with the game, which all involve examination of the built game tree, which is why it is a requirement that the TreeType also is an instance of his GameTree class, which contains operations for examining nodes and the moves that branch off from every node.

    Although it is possible to write your own instance of GameTree, Hagl provides two generic representations of GameTrees: Continuous, and Discrete.  Intermediate nodes have a state and list of outbound edges associated with them, and terminal nodes are payoff nodes, that contain a list of Floats, which are the payoffs for each player in the order the players were passed into the game.

    The game tree edges, as mentioned in the introduction to game theory, must have a concept of payoff associated with them.  In Hagl, this should be represented through 

    Players in Hagl are represented as a simple data type: 

    \begin{verbatim}
        data Player g = forall s. Player Name s (Strategy s g)
              		  | Name ::: Strategy () g
    \end{verbatim}

    Here the Name is just a String, and they must be associated with a strategy, and possibly maintain personal information within the s ~~~~type variable????-- what's the right word here???~~~~~.  

    --struggling to talk intelligently about the strategy type -- how do I explain this:

    \begin{verbatim}
        data StratM s g a = StratM { unS :: StateT s (ExecM g) a }
        type Strategy s g = StratM s g (Move g)
    \end{verbatim}

    The execution of a game happens through the evalGame call:

    \begin{verbatim}
        evalGame :: Game g => g -> [Player g] -> ExecM g a -> IO a
    \end{verbatim}

    ExecM is the game execution monad, but in this context, it's simply a sequence of operations to evaluate.  This can be running through the entire game with the `finish` call, or stepping a number of times (step >> step >> step, etc.). 

    --was weird to try to fit in information about ByPlayer and the other list operations here.  can we just put them in as we need them?

\section{Representational Concepts}
\subsection{Music as a Game}
    Given this general game theoretic framework, how do we then map the fundamental components of a game to aspects of an improvisational performance of two or more musicians? 

    The first critical component is the concept of moves or decisions available to each player, and the rules that specify those moves for all possible game states.  For a musical game, moves take the form of *fixed-duration, time-stamped musical events*. Rather than thinking of a player's full performance as a sequence of notes, we imagine it as a sequence of musical events, that is, a stream of his or her decisions of what and how to play at each time-stamp.

    Events must necessarily be both time-stamped and of a fixed-duration to ensure that players make their decisions of what to play for each event *at the same time*.  Notes in the musical sense are associated with a given duration, but consider a game in which one player begins playing a whole note and the other a quarter: when should we define the next decision point at which the player's can make their next moves?  When the second player is ready to make another move, the first still has three more beats to play.  Thus, we break notes into musical events of fixed-duration and allow a player to choose to extend the pitch of the previous event if he or she desires to play a note of a longer duration.

    In his paper, Hudak notes that the formal model of musical interaction" is limited to controlling an instrument's sound, for the purposes of realizing fundamental parameters such as pitch in addition to more subtle issues of articulation, dynamics and phrasing."  Indeed, in our simple implementation, we focus solely on controlling pitch, ignoring the more subtle parameters and keeping the players� instruments fixed.

    Second, the currency or payoff of the game, used to measure a player's success in the performance, translates to a player's notion of musical aesthetic.  This is a measure of how good the musician considers his own performance sounds, given the actions of all the other players. This preference is unique to each player and may be defined along any axis of the musical design space.

    The realm of musical aesthetic is practically infinite, and a full discussion of music theory of this depth is well beyond the scope of this paper.  However, we decided to model an extremely simple idea of one axis upon which a player may judge the sound of his or her performance, that of pitch intervals - the relative distance on the scale between two notes.

    Finally, each player in a musical game must implement his or her own strategy,which determines how a he or she will move in any given game state.  In a musical improvisational game, a strategy can be as simple as a player adhering to his or her given score, never improvising at all.  Alternatively, a strategy can be more sophisticated, taking into account past or anticipated payoffs, a player's own past moves or those of another player, or some other musical strategy with the only restriction that any move generated by the strategy fall within those allowed by the rules of the game.

    Lastly, for each individual, the goal of the game is to maximize his or her own payoff.  However, we note that unlike zero-sum games, which are purely competitive, a musical game is more cooperative in nature.  Indeed, there exists *some* competition between players, (Hudak imagines two soloists "vying for attention"), however, intuitively, there is a point at which trying to make the other player "sound bad" (i.e. reduce his payoff) has an equally deleterious effect on one's own performance.  Therefore, the best outcome for each player individually is likely to be one in which the sum of their collective payoffs is also maximized.

    For example, in the simplest case, we imagine two horn players soloing simultaneously in a jazz quintet.  Using Hudak's algebraic formulation, the relationship between these two players is a pair of mutually recursive functions:

    \begin{verbatim}
        r1 = instr1(player1 s1 r1 r2)
        r2 = instr2(player2 s2 r1 r2)
    \end{verbatim}

    In terms of the game, *r* refers to each player's realization, the sequence of time-stamped musical events, *s* refers to each player's score (also a sequence of time-stamped musical events) and *player1* and *player2* stand for strategies employed by the given player.  Legal moves at any given point in the game tree are derived from a player's score and realization and payoffs are calculated from realizations and information about the players' preferences.
\subsection{The State Space}
    Size of representation and state space (Note: Paul has a short comment on the computational constraints on strategies on pages 6-7)
    \begin{enumerate}
        \item what is limited as the size of the game/score increases
        \item cheap vs expensive processes
        \item what could be improved?
    \end{enumerate}
\section{Improvise: Our Implementation}
\subsection{Static Components}
    common to all music games)
    \begin{enumerate}
        \item anything data related
            \begin{enumerate}
                \item MusicMv - make sure to explain why we're using Extend Pitch rather than just Extend
                \item SingularScore (Performer)
                \item RealizationState 
                \end{enumerate}
        \item functions related to data -- who, end, registerMove
        \item representation of the tree -- Discrete s mv, gameTree generation function
    \end{enumerate}
\subsection{Dynamic Components}
    \begin{enumerate}
        \item payoff (example)
        \item list of possible moves 
        \item talk about how this is wrapped up in our Improvise data type with the starting state
        \item strategies (example)
    \end{enumerate}
\section{Outcomes}
    \begin{enumerate}
        \item Case study: �Mary Had A Little Lamb�: interval payoffs, playable function based on range, preferences
        \item Case study
    \end{enumerate}
\section{Conclusion}
\section{Future Work}
    \begin{enumerate}
        \item Perhaps your paper can say how such functions would be written and whether it is computationally feasible to evaluate them? (from Norman's feedback)
        \item Continuous game?
    \end{enumerate}


\end{document}
